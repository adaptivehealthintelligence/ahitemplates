% Options for packages loaded elsewhere
\PassOptionsToPackage{unicode}{hyperref}
\PassOptionsToPackage{hyphens}{url}
%
\documentclass[
  11pt,
]{article}
\usepackage{lmodern}
\usepackage{amsmath}
\usepackage{ifxetex,ifluatex}
\ifnum 0\ifxetex 1\fi\ifluatex 1\fi=0 % if pdftex
  \usepackage[T1]{fontenc}
  \usepackage[utf8]{inputenc}
  \usepackage{textcomp} % provide euro and other symbols
  \usepackage{amssymb}
\else % if luatex or xetex
  \usepackage{unicode-math}
  \defaultfontfeatures{Scale=MatchLowercase}
  \defaultfontfeatures[\rmfamily]{Ligatures=TeX,Scale=1}
\fi
% Use upquote if available, for straight quotes in verbatim environments
\IfFileExists{upquote.sty}{\usepackage{upquote}}{}
\IfFileExists{microtype.sty}{% use microtype if available
  \usepackage[]{microtype}
  \UseMicrotypeSet[protrusion]{basicmath} % disable protrusion for tt fonts
}{}
\makeatletter
\@ifundefined{KOMAClassName}{% if non-KOMA class
  \IfFileExists{parskip.sty}{%
    \usepackage{parskip}
  }{% else
    \setlength{\parindent}{0pt}
    \setlength{\parskip}{6pt plus 2pt minus 1pt}}
}{% if KOMA class
  \KOMAoptions{parskip=half}}
\makeatother
\usepackage{xcolor}
\IfFileExists{xurl.sty}{\usepackage{xurl}}{} % add URL line breaks if available
\IfFileExists{bookmark.sty}{\usepackage{bookmark}}{\usepackage{hyperref}}
\hypersetup{
  hidelinks,
  pdfcreator={LaTeX via pandoc}}
\urlstyle{same} % disable monospaced font for URLs
\usepackage[margin=1in]{geometry}
\usepackage{longtable,booktabs}
\usepackage{calc} % for calculating minipage widths
% Correct order of tables after \paragraph or \subparagraph
\usepackage{etoolbox}
\makeatletter
\patchcmd\longtable{\par}{\if@noskipsec\mbox{}\fi\par}{}{}
\makeatother
% Allow footnotes in longtable head/foot
\IfFileExists{footnotehyper.sty}{\usepackage{footnotehyper}}{\usepackage{footnote}}
\makesavenoteenv{longtable}
\usepackage{graphicx}
\makeatletter
\def\maxwidth{\ifdim\Gin@nat@width>\linewidth\linewidth\else\Gin@nat@width\fi}
\def\maxheight{\ifdim\Gin@nat@height>\textheight\textheight\else\Gin@nat@height\fi}
\makeatother
% Scale images if necessary, so that they will not overflow the page
% margins by default, and it is still possible to overwrite the defaults
% using explicit options in \includegraphics[width, height, ...]{}
\setkeys{Gin}{width=\maxwidth,height=\maxheight,keepaspectratio}
% Set default figure placement to htbp
\makeatletter
\def\fps@figure{htbp}
\makeatother
\setlength{\emergencystretch}{3em} % prevent overfull lines
\providecommand{\tightlist}{%
  \setlength{\itemsep}{0pt}\setlength{\parskip}{0pt}}
\setcounter{secnumdepth}{5}
\usepackage{booktabs}
\usepackage{colortbl}
\usepackage{xcolor}
\usepackage{blkarray}
\usepackage{pdflscape}
\usepackage{lastpage}
\usepackage{vhistory}
\usepackage{titling}
\usepackage{fontspec}
\usepackage{setspace}
\usepackage{flafter}
\usepackage{enumitem}
\setlist{nosep}
\newcommand{\blandscape}{\begin{landscape}}
\newcommand{\elandscape}{\end{landscape}}
\usepackage{fancyhdr}
\pagestyle{fancy}
\fancyhf{}
\fancyhead[LO]{PROJECT DSMB Report}
\fancyfoot[LO]{Date: DAY MONTH YEAR}
\fancyfoot[RO]{\thepage\ of \pageref{LastPage}}
\renewcommand{\headrulewidth}{0.4pt}
\renewcommand{\footrulewidth}{0.4pt}
\setlength{\headheight}{14pt}

\setromanfont[Ligatures=TeX]{Latin Modern Roman}
\setsansfont[Ligatures=TeX]{Latin Modern Sans}
\setmonofont[Ligatures=TeX]{Latin Modern Mono}
\setmathfont[Ligatures=TeX]{Latin Modern Math}
\renewcommand{\familydefault}{\sfdefault}

\renewcommand{\arraystretch}{1.2}
\onehalfspacing

\renewcommand{\abstractname}{}
\pagenumbering{gobble}
\usepackage{booktabs}
\usepackage{longtable}
\usepackage{array}
\usepackage{multirow}
\usepackage{wrapfig}
\usepackage{float}
\usepackage{colortbl}
\usepackage{pdflscape}
\usepackage{tabu}
\usepackage{threeparttable}
\usepackage{threeparttablex}
\usepackage[normalem]{ulem}
\usepackage{makecell}
\usepackage{xcolor}
\ifluatex
  \usepackage{selnolig}  % disable illegal ligatures
\fi

\title{\includegraphics[width=\textwidth,height=2in]{logo.jpg}\vspace{2cm}\\
Report for the Data and Safety Monitoring Board\vspace{2cm}}
\usepackage{etoolbox}
\makeatletter
\providecommand{\subtitle}[1]{% add subtitle to \maketitle
  \apptocmd{\@title}{\par {\large #1 \par}}{}{}
}
\makeatother
\subtitle{NAME SHORT - NAME LONG}
\author{}
\date{\vspace{-2.5em}DATE\\
Report Number:}

\begin{document}
\maketitle
\begin{abstract}
\clearpage
\pagenumbering{arabic}

\begin{longtable}{>{}l>{\raggedright\arraybackslash}p{10cm}}
\toprule
\textbf{PROTOCOL TITLE} & xxx\\
\textbf{PROTOCOL NUMBER} & xxx\\
\textbf{PROTOCOL VERSION} & xxx\\
\textbf{\makecell[l]{COORDINATING PRINCIPAL\\INVESTIGATORS}} & xxx\\
\textbf{MEETING DATE} & xxx\\
\addlinespace
\textbf{DATE REPORT} & xxx\\
\textbf{DATA CUTOFF DATE} & xxx\\
\textbf{DATE OF LAST DATA REVIEW} & xxx\\
\textbf{PREPARED BY} & xxx\\
\bottomrule
\end{longtable}
\clearpage
\end{abstract}

\clearpage

\tableofcontents

\clearpage

\hypertarget{executive-summary}{%
\section{Executive Summary}\label{executive-summary}}

\begin{longtable}{>{}l>{\raggedright\arraybackslash}p{10cm}}
\toprule
\textbf{REPORT OVERVIEW} & This report reviews enrolment and safety data available in the study database as of xxxx.Summary tables are provided in the body of the report.Additional tables and figures referenced in the report are provided in the Appendices.\\
\textbf{STUDY SITE STATUS} & Two of the 3 planned study sites have been activated.\\
\textbf{ENROLLMENT STATUS} & xxx subjects have been screened and xxx have been randomised.\\
\textbf{SUBJECT STATUS} & Of the xxx participants randomised, xx are no longer continuing. X withdrew, X died.\\
\textbf{STOPPING RULES} & No stopping rules have been met\\
\addlinespace
\textbf{SAFETY SUMMARY} & No serious adverse events have occurred. xxx adverse events occurred.\\
\textbf{PROTOCOL DEVIATIONS} & xxx deviations have occurred. The reasons relate to...\\
\textbf{QUALITY MANAGEMENT} & Quality checks performed....\\
\bottomrule
\end{longtable}

\clearpage

\hypertarget{protocol-synopsis}{%
\section{Protocol Synopsis}\label{protocol-synopsis}}

\begin{longtable}{>{}l>{\raggedright\arraybackslash}p{10cm}}
\toprule
\textbf{TITLE} & xxx\\
\textbf{BACKGROUND} & xxx\\
\textbf{PRIMARY OUTCOME} & xxx\\
\textbf{SECONDARY OUTCOMES} & \begin{enumerate}[leftmargin=0cm,itemindent=.5cm,labelindent=0pt,nosep]\item Outcome 1\item Outcome 2\end{enumerate}\\
\textbf{STUDY DESIGN} & xxx\\
\addlinespace
\textbf{INTERVENTIONS} & Domain A:\begin{enumerate}\item Control - \item Intervention - \end{enumerate}Domain B:\begin{enumerate}[nolistsep]\item Control - \item Intervention - \end{enumerate}\\
\textbf{STUDY DURATION} & xxx\\
\textbf{SAMPLE SIZE} & xxx\\
\textbf{INCLUSION CRITERIA} & \begin{enumerate}[leftmargin=0cm,itemindent=.5cm,labelindent=0pt,nosep]\item Criteria 1\item Criteria 2\end{enumerate}\\
\textbf{EXCLUSION CRITERIA} & \begin{enumerate}[leftmargin=0cm,itemindent=.5cm,labelindent=0pt,nosep]\item Criteria 1\item Criteria 2\end{enumerate}\\
\addlinespace
\textbf{BLINDING} & xxx\\
\textbf{RANDOMISATION} & xxx\\
\textbf{ANALYSIS} & xxx\\
\bottomrule
\end{longtable}

\clearpage

\hypertarget{report-overview}{%
\section{Report Overview}\label{report-overview}}

\emph{NOTE: THIS TEMPLATE IS DESIGNED TO BE GENERAL PURPOSE FOR EITHER OPEN OR CLOSED REPORTS, BUT SECTIONS SHOULD BE REMOVED OR ADDED WHERE NECESSARY.}
\emph{FOR ALL TABLES, LISTINGS, AND GRAPHS, TWO VERSIONS SHOULD BE PRODUCED: ONE WITH ALL TREATMENT GROUPS POOLED FOR USE IN OPEN REPORTS, AND ONE BY TREATMENT CODE FOR CLOSED REPORTS (PARTIALLY MASKED).}
\emph{IF DMC CHOOSES TO BE UNBLINDED, LISTINGS SHOULD STILL BE PARTIALLY MASKED (IN CASE DOCUMENTS INADVERTENTLY SHARED WITH OTHER BLINDED TRIAL INVESTIGATORS), BUT CODE MAPPINGS CAN BE PROVIDED DURING REPORT REVIEW.}
\emph{PARTIALLY MASKED TREATMENT CODES SHOULD BE CONSISTENT ACROSS REPORTS SO THAT CHANGES BY TREATMENT CAN BE MONITORED.}
\emph{CODE USED TO PRODUCE ALL TABLES, LISTINGS, AND GRAPHS SHOULD BE STORED.}

\emph{GENERALLY, SUMMARIES AND ANALYSES WILL BE BASED ON INTENT-TO-TREAT POPULATION, BUT PER-PROTOCOL/ADHERERS-ONLY ANALYSES MAY ALSO BE OF INTEREST.}
\emph{ALL MOCK TEXT/FIGURES ARE EXAMPLES ONLY AND SHOULD BE TAILORED TO THE TRIAL AND DMC REQUIREMENTS.}
\emph{THE DMC CAN RECOMMEND ADDITIONAL DATA DISPLAYS OR CHANGES TO/REMOVAL OF DATA DISPLAYS AS REQUIRED FOR FUTURE REPORTS AS TRIAL PROGRESSES.}

The purpose of this report is to review cumulative enrolment, data quality, and safety data for the subjects enrolled in the ASCOT study.
This report reflects data from the study database as of {[}DATE{]}.
Within the body of the report are summary tables of enrolment, demographic characteristics, and adverse events.
Additional tables, listings, and figures referenced in this report are provided in Appendices.
There have been {[}Insert number of meetings{]} DSMB meetings for this study, and the last review was on {[}DATE{]}.

\emph{Summarise what the recommendations regarding the trial were following the last DSMB meeting.}

Readers of this report are asked to maintain the confidentiality of the information provided in this report.

\clearpage

\hypertarget{response-to-dsmb-recommendations}{%
\section{Response to DSMB Recommendations}\label{response-to-dsmb-recommendations}}

\emph{Identify DSMB recommendations/requests from the last meeting and clarify how those requests have been handled in the report and/or elsewhere.}
\emph{If this is the first DSMB meeting for this protocol or no previous recommendations/requests were made, indicate as such in this section.}
\emph{Doing so will provide a future reminder to the author who is likely to use the previous report as a starting point for the subsequent report.}

\clearpage

\hypertarget{enrolment-status}{%
\section{Enrolment Status}\label{enrolment-status}}

\emph{Describe enrolment and provide a summary table.}
\emph{If stratified randomisation, may look at enrolment by strata.}
\emph{Provide enrolment statistics by site if the study involves multiple sites.}
\emph{If the study is still enrolling, provide the subject accrual target and estimated time to completion of enrolment.}
\emph{A figure showing expected/planned versus actual enrolment may be helpful (see example on next page).}

\emph{If closed report, may stratify enrolment by allocated intervention.}

FLOWCHART

\begin{figure}
\centering
\includegraphics{skeleton_files/figure-latex/unnamed-chunk-4-1.pdf}
\caption{\label{fig:unnamed-chunk-4}Cumulative number of trial participants randomised over time.}
\end{figure}

\begin{table}

\caption{\label{tab:unnamed-chunk-5}Participant enrolment by study site.}
\centering
\begin{tabular}[t]{l>{\raggedright\arraybackslash}p{2cm}>{\raggedleft\arraybackslash}p{2cm}>{\raggedleft\arraybackslash}p{2cm}>{\raggedleft\arraybackslash}p{2cm}>{\raggedleft\arraybackslash}p{2cm}}
\toprule
Site & Participation Start Date & Screened & Enrolled & Completed Follow-up & Enrolled per month\\
\midrule
S001 - Name 1 & xxxx-xx-xx & xx & xx & xx & xx\\
S002 - Name 2 & xxxx-xx-xx & xx & xx & xx & xx\\
S003 - Name 3 & xxxx-xx-xx & xx & xx & xx & xx\\
\bottomrule
\end{tabular}
\end{table}

\begin{figure}
\centering
\includegraphics{skeleton_files/figure-latex/unnamed-chunk-6-1.pdf}
\caption{\label{fig:unnamed-chunk-6}Cumulative number of trial participants randomised over time by site.}
\end{figure}

\clearpage

\hypertarget{participant-status}{%
\section{Participant Status}\label{participant-status}}

\emph{Describe where patients are in the study in relation to major milestones, such as the number of subjects who have completed day 1, the number on treatment, completed day 28, and the final study visit.}
\emph{A summary table providing the study milestones and the number of subjects who have completed those milestones is recommended.}
\emph{Also, provide the number of subjects who were terminated and the reason for their termination, such as voluntary withdrawal, death, lost to follow-up, adverse event, or completed the protocol.}
\emph{A summary table of subject disposition is also recommended.}

\emph{It is important to distinguish between subjects who withdrew early from the study and those who discontinued treatment but may or may not still be followed-up.}

\clearpage

\hypertarget{demographics}{%
\section{Demographics}\label{demographics}}

\emph{Describe the demographic characteristics (age, gender and ethnicity) and key baseline characteristics of enrolled subjects (if appropriate). Provide a summary table or a listing of the data. Listings are preferable over summary tables if only a few subjects have been enrolled. However, avoid listing any information that could potentially lead to the identification of a participant.}

\clearpage

\hypertarget{safety}{%
\section{Safety}\label{safety}}

\emph{Note whether AEs/SAEs are only reported for those specified in relevant protocol.}

\hypertarget{adverse-events}{%
\subsection{Adverse Events}\label{adverse-events}}

\emph{Summarize or list the adverse events (AEs) that have occurred since the previous DSMB report and over the course of the trial.}
\emph{Provide information on severity (e.g.~none, mild, moderate, severe, life-threatening) and relatedness to protocol treatment and study procedures (see an example of a summary table below) as per protocol.}
\emph{Ensure that categories summarized match those in the protocol, for instance unrelated/related vs.~3 category (unrelated, possible, related) vs.~5 category delineation (unrelated, unlikely, possible, probable, definite).}
\emph{State standardised dictionary coding if being used for recording adverse events (e.g.~MedDRA).}
\emph{Other details may be provided, such as other concomitant medications, additional protocol treatment details, and contributing factors.}

\emph{In the closed report, listings and summaries may be stratified by domain and/or intervention.}.
\emph{If a participant may receive a combination of treatments (e.g.~factorial trial) note that the same AE may appear under multiple treatments in data summaries, or summarise AEs by combined treatment regimen.}

\begin{table}

\caption{\label{tab:unnamed-chunk-7}Summary of adverse events (AEs). If exposure sufficient, might present incidence rates per participant and per xxx exposure.}
\centering
\begin{tabular}[t]{lrrr}
\toprule
  & Treatment A & Treatment B & Total\\
\midrule
Total & 15 & 16 & 31\\
Exposure (participant days on study) & 420 & 448 & 868\\
\addlinespace[0.3em]
\multicolumn{4}{l}{\textbf{AEs by severity}}\\
\hspace{1em}Mild & 10 & 10 & 20\\
\hspace{1em}Moderate & 5 & 5 & 10\\
\hspace{1em}Severe & 0 & 1 & 1\\
\addlinespace[0.3em]
\multicolumn{4}{l}{\textbf{AEs by relatedness}}\\
\hspace{1em}Unrelated & 10 & 10 & 20\\
\hspace{1em}Unlikely & 5 & 5 & 10\\
\hspace{1em}Possible & 0 & 0 & 0\\
\hspace{1em}Probable & 0 & 0 & 0\\
\hspace{1em}Definite & 0 & 1 & 1\\
\bottomrule
\end{tabular}
\end{table}

\begin{table}

\caption{\label{tab:unnamed-chunk-8}Comparison of adverse events by preferred term.}
\centering
\begin{tabular}[t]{lrrrr}
\toprule
\multicolumn{1}{c}{Preferred Term} & \multicolumn{1}{c}{\makecell[c]{Trt A\\(n = 214)}} & \multicolumn{1}{c}{\makecell[c]{Trt B\\(n = 223)}} & \multicolumn{1}{c}{Relative Risk (95\% CI)} & \multicolumn{1}{c}{p-value}\\
\midrule
Term 1 & 10 (4.7) & 15 (6.73) & 0.70 (0.32, 1.51) & 0.36\\
Term 2 & 63 (29.4) & 30 (13.5) & 2.19 (1.48, 3.24) & <0.01\\
\bottomrule
\multicolumn{5}{l}{\textsuperscript{} n: subjects each treatment}\\
\multicolumn{5}{l}{\textsuperscript{} CI: confidence interval}\\
\multicolumn{5}{l}{\textsuperscript{} p-value: Chi-square test}\\
\end{tabular}
\end{table}

\begin{landscape}\begingroup\fontsize{8}{10}\selectfont

\begin{longtable}[t]{>{\raggedright\arraybackslash}p{2cm}>{\raggedright\arraybackslash}p{1.5cm}>{\raggedright\arraybackslash}p{1.5cm}>{\raggedright\arraybackslash}p{1.5cm}>{\raggedright\arraybackslash}p{4cm}>{\raggedright\arraybackslash}p{2cm}>{\raggedright\arraybackslash}p{2cm}>{\raggedright\arraybackslash}p{2cm}>{\raggedright\arraybackslash}p{2cm}}
\caption{\label{tab:unnamed-chunk-9}Adverse events listings. AE's occurring snice last DSMB report highlighted and SAE's italicised.}\\
\toprule
Participant ID & Diagnosis & Event & Coding & Dates & Serious & Severity & Outcome & Relatedness\\
\midrule
Pxxxxx & xx & xx & xx & xxxx-xx-xx - xxxx-xx-xx & No & Mild & Unresolved & Unrelated\\
\cellcolor{yellow}{Pxxxxx} & \cellcolor{yellow}{xx} & \cellcolor{yellow}{xx} & \cellcolor{yellow}{xx} & \cellcolor{yellow}{xxxx-xx-xx - xxxx-xx-xx} & \cellcolor{yellow}{No} & \cellcolor{yellow}{Moderate} & \cellcolor{yellow}{Resolved} & \cellcolor{yellow}{Possible}\\
\cellcolor{yellow}{\em{Pxxxxx}} & \cellcolor{yellow}{\em{xx}} & \cellcolor{yellow}{\em{xx}} & \cellcolor{yellow}{\em{xx}} & \cellcolor{yellow}{\em{xxxx-xx-xx - xxxx-xx-xx}} & \cellcolor{yellow}{\em{Yes}} & \cellcolor{yellow}{\em{Severe}} & \cellcolor{yellow}{\em{Resolved}} & \cellcolor{yellow}{\em{Definite}}\\
\bottomrule
\end{longtable}
\endgroup{}
\end{landscape}

\hypertarget{serious-adverse-events}{%
\subsection{Serious Adverse Events}\label{serious-adverse-events}}

\emph{Summarize or list all serious adverse events (SAEs) that have occurred since the previous DSMB report and over the course of the trial.}

\emph{In the closed report, listings will be stratified by domain and/or intervention.}.

\begin{landscape}\begingroup\fontsize{8}{10}\selectfont

\begin{longtable}[t]{>{\raggedright\arraybackslash}p{2cm}>{\raggedright\arraybackslash}p{1.5cm}>{\raggedright\arraybackslash}p{1.5cm}>{\raggedright\arraybackslash}p{1.5cm}>{\raggedright\arraybackslash}p{4cm}>{\raggedright\arraybackslash}p{2cm}>{\raggedright\arraybackslash}p{2cm}>{\raggedright\arraybackslash}p{2cm}}
\caption{\label{tab:unnamed-chunk-10}Serious adverse events listings. SAE's occurring snice last DSMB report highlighted.}\\
\toprule
Participant ID & Diagnosis & Event & Coding & Dates & Severity & Outcome & Relatedness\\
\midrule
\cellcolor{yellow}{Pxxxxx} & \cellcolor{yellow}{xx} & \cellcolor{yellow}{xx} & \cellcolor{yellow}{xx} & \cellcolor{yellow}{xxxx-xx-xx - xxxx-xx-xx} & \cellcolor{yellow}{Severe} & \cellcolor{yellow}{Resolved} & \cellcolor{yellow}{Definite}\\
\bottomrule
\end{longtable}
\endgroup{}
\end{landscape}

\clearpage

\hypertarget{protocol-deviations}{%
\section{Protocol Deviations}\label{protocol-deviations}}

\emph{Summarize or list protocol deviations that have occurred since the previous DSMB report and over the course of the study.}
\emph{Additionally, report serious breaches of GCP or protocol, e.g.~breach likely to affect safety of participant, reliability of trial generated data.}

The following protocol deviations will be reported:
- missed visits
- incorrect dosing or wrongly prescribed study drug or treatment
- missing primary outcome

\clearpage

\hypertarget{effectiveness}{%
\section{Effectiveness}\label{effectiveness}}

\hypertarget{primary-outcome}{%
\subsection{Primary Outcome}\label{primary-outcome}}

\hypertarget{secondary-outcomes}{%
\subsection{Secondary Outcomes}\label{secondary-outcomes}}

\clearpage

\hypertarget{appendix-a-additional-figures}{%
\section*{Appendix A: Additional Figures}\label{appendix-a-additional-figures}}
\addcontentsline{toc}{section}{Appendix A: Additional Figures}

\clearpage

\hypertarget{appendix-b-additional-documentation}{%
\section*{Appendix B: Additional Documentation}\label{appendix-b-additional-documentation}}
\addcontentsline{toc}{section}{Appendix B: Additional Documentation}

\end{document}
